% Chapter 1

\chapter{Introduction} % Main chapter title

\label{Chapter1} % For referencing the chapter elsewhere, use \ref{Chapter1} 

\lhead{Chapter 1. \emph{Introduction}} % This is for the header on each page - perhaps a shortened title

%----------------------------------------------------------------------------------------

Optical character recognition(OCR) is a well known process for converting text images to machine
editable text format. During past few decades significant research work is reported in the OCR area.
In English there are many  OCR applications are available . Apart from English there
is significant amount of research have been done for languages like Chinese  and Japanese.
OCR gained so much research interest because of its potential applications in post offices, banks and
defense organizations. Other applications involve reading aid for the blind, preserving old/historical
documents in electronic format, library cataloging, automatic reading for sorting bank cheques and
applications in natural language processing(NLP) area.

OCR implemented for English language can not be applied for Indian languages as the single
character formation in the Indian language can be either simple character formed by single vowel
or consonant or compound character formed by combination of the vowel and consonants.

\textbf{Motivation:}The research in OCR area all other languages like English, Chinese and Japanese
are far a head compared to the research done in Indian languages. In particular the research done
in OCR for Telugu language is not significant. We want to develop an OCR which can mitigate the error 
due to segmentation improve the accuracy. We want to
digitalize the old/historical documents of Telugu language which are available in digital library of
India.

Rest of the report is organized as follows.In Chapter 2 we will describe the summary of the segmentation 
algorithm. In chapter 3 we will describe architecture of neural network we  experimented for
solving problem. In chapter 4 we will describe the conclusion and future work.